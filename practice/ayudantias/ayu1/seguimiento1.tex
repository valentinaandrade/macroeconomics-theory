% This LaTeX was auto-generated from MATLAB code.
% To make changes, update the MATLAB code and export to LaTeX again.

\documentclass{article}

\usepackage[utf8]{inputenc}
\usepackage[T1]{fontenc}
\usepackage{lmodern}
\usepackage{graphicx}
\usepackage{color}
\usepackage{hyperref}
\usepackage{amsmath}
\usepackage{amsfonts}
\usepackage{epstopdf}
\usepackage[table]{xcolor}
\usepackage{matlab}

\sloppy
\epstopdfsetup{outdir=./}
\graphicspath{ {./seguimiento1_images/} }

\begin{document}

\matlabtitle{Seguimiento N° 1}

\begin{par}
\begin{flushleft}
\textbf{Teoría Macroeconómica I }
\end{flushleft}
\end{par}

\begin{par}
\begin{flushleft}
\textbf{Estudiante}: Valentina Andrade de la Horra                 \textbf{Fecha}:  11 de marzo 2022
\end{flushleft}
\end{par}

\begin{par}
\begin{flushleft}
\textbf{Profesor}: Alexandre Janiak    Ayudante: Pablo Vega
\end{flushleft}
\end{par}

\matlabheading{3.1 Introducción a Matlab}

\begin{par}
\begin{flushleft}
He sido contratada por una empresa encargada de llevar las estadísticas de la cadena de supermercados \textit{Matjobs, }por lo cual se  me encomendaron la tarea de estudiar el \textbf{comportamiento de los clientes} dentro del supermercado, específicamente de los tiempos en ingresos, búsqueda de productos y el pago. 
\end{flushleft}
\end{par}

\begin{par}
\begin{flushleft}
A partir de una muestra de 1.000 personas que asistieron todos los días del mes de abril (30 días) a \textit{Matjobs,} he logrado obtener la distribución de los principales flujos del supermercado (medidas en minutos).
\end{flushleft}
\end{par}

\begin{par}
\begin{flushleft}
0. Limpiar espacio de trabajo
\end{flushleft}
\end{par}

\begin{matlabcode}
clc
clear
\end{matlabcode}

\begin{par}
\begin{flushleft}
1. Parámetros 
\end{flushleft}
\end{par}

\begin{par}
\begin{flushleft}
Prealocaremos algunas variables. Luego contruiremos la distribución de las variables de interés.
\end{flushleft}
\end{par}

\begin{matlabcode}
i = 1000;
t = 30;
r = i*t;
c = 1;
\end{matlabcode}

\begin{par}
\begin{flushleft}
\textbf{Ingreso (I)}
\end{flushleft}
\end{par}

\begin{par}
\begin{flushleft}
I \textasciitilde{} exp(2)
\end{flushleft}
\end{par}

\begin{matlabcode}
I = exp(2) + rand(r,c)
\end{matlabcode}
\begin{matlaboutput}
I = 30000x1    
    8.2902
    7.6570
    7.4332
    7.4030
    7.5826
    8.0917
    8.1675
    7.3900
    8.0793
    8.3506

\end{matlaboutput}

\begin{par}
\begin{flushleft}
\textbf{Búsqueda (B)}
\end{flushleft}
\end{par}

\begin{par}
\begin{flushleft}
B \textasciitilde{} N(0.3,1)
\end{flushleft}
\end{par}

\begin{matlabcode}
B = randn(r,c)*0.3+1
\end{matlabcode}
\begin{matlaboutput}
B = 30000x1    
    0.7944
    0.8308
    1.1768
    1.5342
    1.4045
    1.0589
    0.8879
    0.8224
    0.8982
    1.4844

\end{matlaboutput}

\begin{par}
\begin{flushleft}
\textbf{Pago (P)}
\end{flushleft}
\end{par}

\begin{par}
\begin{flushleft}
P \textasciitilde{} N(0.15,0.2)
\end{flushleft}
\end{par}

\begin{matlabcode}
P = randn(r,c)*0.15+0.2
\end{matlabcode}
\begin{matlaboutput}
P = 30000x1    
    0.3957
    0.1593
    0.2059
   -0.2176
    0.3037
    0.2292
    0.1420
    0.1293
    0.1475
    0.0107

\end{matlaboutput}

\begin{par}
\begin{flushleft}
Por definición, se establece que la suma de los tiempos de \textit{ingreso, búsqueda y pago} corresponde al tiempo total dentro del supermercado. 
\end{flushleft}
\end{par}

\matlabheading{El ejercicio incluye que }

\begin{par}
\begin{flushleft}
(a) Utilizando loops he creado una matriz TL que contiene como fila a cada cliente y como columna el tiempo total de compra del mismo cliente para cada día del mes de abril. 
\end{flushleft}
\end{par}

\begin{matlabcode}
% Preallocate matrix
TL = zeros(r, 4); 
T = I + B + P; % Crear fila total
for col = 1:4
    vars = [I B P T];
    TL(:,col) = vars(:,col);
end

\end{matlabcode}

\begin{par}
\begin{flushleft}
(b) Obtenga la misma matriz del ítem anterior pero de forma matricial. Nombre esta matriz TM. 
\end{flushleft}
\end{par}

\begin{matlabcode}
TM = [I B P T]
\end{matlabcode}
\begin{matlaboutput}
TM = 30000x4    
    8.2902    0.7944    0.3957    9.4803
    7.6570    0.8308    0.1593    8.6471
    7.4332    1.1768    0.2059    8.8160
    7.4030    1.5342   -0.2176    8.7196
    7.5826    1.4045    0.3037    9.2908
    8.0917    1.0589    0.2292    9.3799
    8.1675    0.8879    0.1420    9.1974
    7.3900    0.8224    0.1293    8.3417
    8.0793    0.8982    0.1475    9.1250
    8.3506    1.4844    0.0107    9.8458

\end{matlaboutput}

\begin{par}
\begin{flushleft}
Comprobamos que ambas matrices son iguales
\end{flushleft}
\end{par}

\begin{matlabcode}
 
if isequal(TL,TM) == 1
    disp('Las matrices son iguales')
else
    disp('No son iguales las matrices')
end
\end{matlabcode}
\begin{matlaboutput}
Las matrices son iguales
\end{matlaboutput}

\begin{par}
\begin{flushleft}
(c) Identifique al individuo que más tiempo tardó en completar el proceso de compra y compute dicho tiempo
\end{flushleft}
\end{par}

\begin{matlabcode}
max(TM(:,4))
\end{matlabcode}
\begin{matlaboutput}
ans = 10.7353
\end{matlaboutput}

\begin{par}
\begin{flushleft}
(d) Identifique al individuo que menos tiempo tardó en completar el proceso de compra y compute dicho tiempo
\end{flushleft}
\end{par}

\begin{matlabcode}
disp(['El individuo que tardó más, tardó en total ',num2str(max(TM(:,4))),' minutos. Su posición en la muestra es ', num2str(find(TM == max(TM(:,4))))])   
\end{matlabcode}
\begin{matlaboutput}
El individuo que tardó más, tardó en total 10.7353 minutos. Su posición en la muestra es 100347
\end{matlaboutput}

\begin{par}
\begin{flushleft}
(e) Cree la función \textbf{test} que recibe como inputs el número de personas y la cantidad de días que asistió a comprar y que entrega como \textbf{outputs}
\end{flushleft}
\end{par}

\begin{matlabcode}
clc
clear
\end{matlabcode}

\begin{par}
\begin{flushleft}
La función \textbf{test.m permite calcular lo siguiente. }Para el punto 1, 2 y 4, se puede extraer como resultante. Mientras que para 3 y 5 aparece un output de resumen de esta información (junto con 2 y 4 respectivamente)    
\end{flushleft}
\end{par}

\begin{par}
\begin{flushleft}
(1) un vector columna VL que contiene el tiempo total de compra de cada persona obtenido a través de un loop
\end{flushleft}
\end{par}

\begin{par}
\begin{flushleft}
(2) la persona que más pasó tiempo comprando
\end{flushleft}
\end{par}

\begin{par}
\begin{flushleft}
(3) el tiempo asociado a esa persona
\end{flushleft}
\end{par}

\begin{par}
\begin{flushleft}
(4) la persona que menos tiempo pasó comprando
\end{flushleft}
\end{par}

\begin{par}
\begin{flushleft}
(5) el tiempo asociado a esa persona 
\end{flushleft}
\end{par}

\begin{par}
\begin{flushleft}
Si se quiere saber como ocupar \textbf{test} deben poner help test. Se ha creado una ayuda para poder ocuparla. Imaginemos que al supermercado asistieron 2 personas durante 4 días.
\end{flushleft}
\end{par}

\begin{matlabcode}
test(2,4)
\end{matlabcode}
\begin{matlaboutput}
El individuo que tardó más, tardó en total 9.6205 minutos. Su posición en la muestra es la posición 6
El individuo que tardó memnos, tardó en total 8.4622 minutos. Su posición en la muestra es la posición 5
ans = 8x1    
    9.1062
    9.3308
    9.3709
    8.5929
    8.4622
    9.6205
    8.7142
    9.0103

\end{matlaboutput}

\begin{par}
\begin{flushleft}
Como se puede obtener, tenemos como output una guía que nos dice al inicio los mínimos y maximos, y las posiciones. Además nos muestra la matriz VL contruida. Para poder guardar estos objetos en el espacio de trabajo lo hacemos de la siguiente manera
\end{flushleft}
\end{par}

\begin{par}
\begin{flushleft}
\textbf{[VL,maxT,minT] = test(i,t)}
\end{flushleft}
\end{par}

\begin{par}
\begin{flushleft}
Por ejemplo, 
\end{flushleft}
\end{par}

\begin{matlabcode}
[VL,maxT,minT] = test(2,4)
\end{matlabcode}
\begin{matlaboutput}
El individuo que tardó más, tardó en total 9.481 minutos. Su posición en la muestra es la posición 3
El individuo que tardó memnos, tardó en total 8.3629 minutos. Su posición en la muestra es la posición 7
VL = 8x1    
    8.4098
    8.6108
    9.4810
    9.1623
    8.5360
    8.8300
    8.3629
    9.2468

maxT = 9.4810
minT = 8.3629
\end{matlaboutput}

\end{document}
